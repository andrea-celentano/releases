% preambolo per doppia compilazione HTML/PDF
\ifx\pdfoutput\undefined      % compilazione htlatex
\documentclass{article}
\DeclareGraphicsExtensions{.png, .gif, .jpg}
\newcommand{\href}[2]{\Link[#1]{}{} #2 \EndLink}
\newcommand{\hypertarget}[2]{\Link[]{}{#1} #2 \EndLink}
\newcommand{\hyperlink}[2]{\Link[]{#1}{} #2 \EndLink}
\else                         % compilazione pdflatex
\documentclass{article}
\usepackage{graphicx}
\usepackage{listings}
\usepackage{fancyhdr}
\usepackage{wrapfig}
\usepackage{multirow}
\usepackage{lscape}
\usepackage{amssymb,amsmath}
\pdfpagewidth 8.5in
\pdfpageheight 11in
\setlength\textwidth{5.7in}
\setlength\textheight{8.1in}
\setlength\oddsidemargin{0in}
\setlength\evensidemargin{0in}
\setlength\topmargin{-0.6in}
\setlength\footskip{0.6in}
\setlength\headsep{0.6in}
\usepackage[hyperindex]{hyperref}
\newcommand{\percent}{\,^0\!/_0}
 \hypersetup{
    bookmarks=true,         % show bookmarks bar?
    unicode=false,          % non-Latin characters in Acrobat’s bookmarks
    pdftoolbar=true,        % show Acrobat’s toolbar?
    pdfmenubar=true,        % show Acrobat’s menu?
    pdffitwindow=true,      % page fit to window when opened
    pdfauthor={Maurizio},   % author
    pdfsubject={Ungaro},    % subject of the document
    pdfnewwindow=true,      % links in new window
    colorlinks=true,        % false: boxed links; true: colored links
    linkcolor=black,        % color of internal links
    citecolor=blue,         % color of links to bibliography
    filecolor=magenta,      % color of file links
    urlcolor=blue           % color of external links
}
\fi


\begin{document}
\pagestyle{fancy}
\renewcommand{\sectionmark}[1]{\markright{\slshape \thesection\ #1}{}}
\fancyhead[R]{\bf\rightmark} 
\fancyhead[L]{gemc}
\fancyfoot[R]{ \sl M. Ungaro}
\fancyfoot[L]{ \sl UCONN/JLAB}

\title{\large {\bf \huge GE}ant4 {\bf \huge M}onte {\bf \huge C}arlo documentation}
 \author{M. Ungaro}
\maketitle
\vspace{1cm}
\abstract{The first part of the document provides a tutorial on how to run gemc.
					The second part describes how to insert/modify new/existing geometry,
					hit processes, and bank output information.
					The third part is for advanced users and describe in full details
					the features of gemc.}
\newpage
\tableofcontents


\newpage
\noindent
\vspace{9 cm}
\begin{center}
Even if there is nothing left but desert waste,
I should want to be there, to see dune folding upon dune.
If there is but one lamp left in all the world,
I'd want to watch its flame.
\end{center}

\begin{flushright}
\it Marius
\end{flushright}



\newpage
\noindent
\vspace{8 cm}
\begin{center}
It has only to do with the respect with which we regard one another,
the dignity of men, our love of culture.
It has to do with: Are we good painters, good sculptors, great poets?
I mean all the things we really venerate in our country and are
patriotic about. It has nothing to do directly with defending our
country except to make it worth defending.
\end{center}

\begin{flushright}
\it R.R. Wilson
\end{flushright}

\newpage




\section{Introduction}
gemc is a framework based on Geant4 libraries in order to run MonteCarlo simulations of particle detectors.



In the first part of this book I'll assume you have gemc up and running.



\section{Running gemc}


In this section you'll learn:
\begin{itemize}
\item how to select the detectors and geometry parts to include in the simulation
\item how to use the gemc particles generator, and luminosity feature
\item how to use the LUND format as event generator
\item choose and read the output format
\end{itemize}

\subsection{}










\begin{thebibliography}{mybib}
 \bibitem {bib:valeri_vertex}   {Valeri Koubarovski},     {\it Private Communication.}
\end{thebibliography}


\end{document}








